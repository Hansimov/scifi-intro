ocw.nthu.edu.tw/videosite/index.php?op=watch&id={}&filename=640_480_768.MP4&type=download

第1讲 课程介绍与上课方式说明
    237
    L01_A 介绍上课与评分方式,警告同学须确实出席与交作业

第2讲 电影《梦游交易所》欣赏与相关概念
    240 241
    L02_A 上课规矩说明、处理学生选课问题,介绍本学期课程,与同学互动,讨论「什么是科幻?」
    L02_B 从《梦游交易所》来谈各种科幻主题

第3讲 何谓科幻?科幻概论之总论篇
    242 243 244
    L03_A 上课与评分方式说明、本学期课程单元介绍
    L03_B 科幻滥觞与命名、科学幻想与幻想科学的差异
    L03_C 科幻文学与映像、科幻主题的常见元素

第4讲 科幻的10+1堂入门必修课之「人」、「空」、「时」、「灾」、「战」、「政」
    245 246 247
    L04_A 
        「人」(Monster&Aliens)
    L04_B
        「空」(Space Opera)
        「时」(Time Travel)
    L04_C
        「灾」(Tremendous Powers)
        「战」(Super Weapons)
        「政」(Dystopia)

第5讲 科幻的10 + 1堂入门必修课之「力」,「界」,「创」,「神」,「变」
    248 249
    L05_A
        「力」(Super Beings)
        「界」(Cyber​​ Millenium)
    L05_B
        「创」(Creations)
        「神」(Transshumanism)
        「变」(PKD)

第6讲 古典科幻篇
    250 251
    L06_A 讨论上周播放的影片《科学怪人》,并将上周未看完的影片播完。
    L06_B 
        谈准科幻的记载,科幻的起点与定义谈到玛丽雪莱的《科学怪人》。
        接着谈Jules Verne(1928-2905)、HG Wells(1866-1946)两人的作品。
        最后整理西方古典科幻简史。

第7讲 社会科幻篇与电影《重装任务》
    252
    L07_A 社会科幻的反乌托邦特色与影片《重装任务》介绍

第8讲 社会科幻篇与电影《千钧一发》
    253
    L08_A 从社会制约谈社会科幻

第9讲 社会科幻篇与反乌托邦
    378 383
    L09_A 完整介绍社会科幻,延续统整前几周内容,谈乌托邦与反乌托邦
    L09_B 介绍社会科幻重要作品:《我们》、《美丽新世界》、《1984》、《华氏451度》

第10讲 美国科幻黄金时期与电影《星战毁灭者》
    380
    L10_A 概谈美国科幻黄金时期与电影《惊爆银河系》(Galaxy Quest)、《星战毁灭者》(Mars Attacks)

第11讲 英语科幻三大家
    384 385 386
    L11_A 美国科幻黄金时期与英语科幻三大奖的介绍
    L11_B 英语科幻三大家:罗伯特.海来茵(Robert. A. Heinlein )
    L11_C 英语科幻三大家:亚瑟.克拉克(Arthur C. Clarke)、以撒.艾西莫夫(Isaac Asimov)

第12讲 英语科幻三大家之艾西莫夫与电影《魔鬼总动员》
    387 388
    L12_A 艾西莫夫的银河系列与基地系列
    L12_B 艾西莫夫的基地系列与法兰克.赫伯特的沙丘系列
    (本讲课程投影片与上一讲相同,延续相同主题)

第13讲 菲利浦.狄克和他的异想世界
    389 390
    L13_A 菲利浦.狄克的生平
    L13_B 菲利浦.狄克原著小说改编电影

第14讲 菲利浦.狄克与电影《异次元骇客》
    473
    L14_A 菲利浦.狄克重要作品介绍:《银翼杀手》、《魔鬼总动员》、《异形终结》、《强殖入侵》、《关键报告》、《记忆裂痕》、《心机扫描》、《关键下一秒》

第15讲 Cyber​​punk 驭电飞行
    465 466
    L15_A 何谓cyber punk、谈科技对人类社会的影响、cyberpunk source book
    L15_B cyperpunk诞生背景及常见题材、《Tron》、《Neuromancer》、Stempunk

第16讲 Cyber​​punk与电影《Tron》
    467 468
    L16_A 介绍电影《Tron》
    L16_B 《Tron》的后续发展与概念、cyberpunk游戏介绍

第17讲 骇客任务影片欣赏
    472
    L17_A 

第18讲 日本的幻想科学
    471
    L18_A 日本的幻想科学 
        § 透过“ 民族历史经验、传播形式、代表特征” 简介日本幻想科学  
        § 讲述日本幻想科学中的3大代表文化与相关作品
        § 从文化观点比较美日幻想科学